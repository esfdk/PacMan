\section{Results}
The results of the Behaviour Tree controller submitted to the Initial Ms. PacMan competition\cite{competition} can seen in table \ref{table-compResults}. 
\begin{table}[h]
\caption{Competition controller results}
\begin{center}
\begin{tabular}{|c|c|c|c|c|c|}
\hline
\multicolumn{1}{|c|}{\raisebox{-0.50ex}[0cm][0cm]{\!Ghost\!}}
& \multicolumn{1}{|c|}{Average}
& \multicolumn{1}{|c|}{Max}
& \multicolumn{1}{|c|}{Min} 
& \multicolumn{1}{|c|}{Standard}
& \multicolumn{1}{|c|}{Relative}\\
controller & score & score & score & deviation & std dev\\ \hline
Legacy2 &$4624.43$&$20800$&$1620$&$2864.87$ & $61.95\%$\\ \hline
Starter &$10752.05$&$34550$&$1970$&$5622.16$ & $52.29\%$\\ \hline
Aggressive &$16442.54$&$71240$&$2750$&$11856.77$ & 62.12\% \\ \hline
Random &$53806.59$&$66790$&$2570$&$10507.85$ & $19.53\%$\\ \hline
\end{tabular}
\label{table-compResults}
\end{center}
\end{table}

The relation between the average scores of the controller against the different ghost controllers are to be expected. Both the Legacy2 and Starter ghost controllers are harder to predict as they have a balance between the all-out aggressive nature of the Aggressive Ghosts and the completely stochastic nature of the Random Ghosts. Because the controller is able to make immediate turnarounds due to the "Flee" behaviour, the Random Ghosts will perform poorly against this controller, as PacMan will rarely be boxed in by them. 

The relative standard deviation of the results show much of the same story, except for the results of the aggressive ghost controllers. If the behaviour of the ghost controllers is entirely deterministic, the standard deviation of the score should be low. It is likely, however, that minuscule timings make the PacMan and/or Ghost controllers make different choices in different playthroughs.

\subsection{Neural Network controller}
\begin{table}[h]
\caption{Neural Network controller results}
\begin{center}
\begin{tabular}{|c|c|c|c|c|c|}
\hline
\multicolumn{1}{|c|}{\raisebox{-0.50ex}[0cm][0cm]{\!Ghost\!}}
& \multicolumn{1}{|c|}{Average}
& \multicolumn{1}{|c|}{Max}
& \multicolumn{1}{|c|}{Min} 
& \multicolumn{1}{|c|}{Standard}
& \multicolumn{1}{|c|}{Relative}\\
controller & score & score & score & deviation & std dev\\ \hline
Legacy2 &  $120.0$  &  $120$  &  $120$ & $0.0$ & 0\%\\ \hline
Starter &  $120.0$  &  $120$  &  $120$ & $0.0$ & 0\%\\ \hline
Aggressive &  $120.0$  &  $120$  &  $120$ & $0.0$ & 0\%\\ \hline
Random &  $181.99$  &  $820$  &  $70$ & $83.512$ & 45.89\%\\ \hline
\end{tabular}
\label{table-nnResults}
\end{center}
\end{table}

The performance of the neural network controller was abysmal. As shown by table \ref{table-nnResults}, no higher score than 70 was achieved. This is likely due to a problem with the neural network being incapable of classifying any of it's training (or test) tuples correctly. Due to the nature of the simple inputs, the network attempts to classify the current game state into an output. As noted in \ref{A-NN}, a variety of topologies consisting of different input combinations did little to change the performance of the controller. The reason the controller performs better against the Random ghosts is purely due to the stochastic nature of the ghost controller.

The results of testing with the "AND" and "XOR" truth tables showed that both the neural network and backpropagation algorithms were fully functional, so the problem most certainly lies with either the way in which the neural network is used or the input given to the neural network. Worthy of note is also that each epoch of training the neural network takes much longer for a PacMan controller than for the "AND" and "XOR" truth tables.
\subsubsection{Suggested improvements}
Due to the lack of any prediction or simulation of future game states, the neural network is only capable of considering the current game state and as such is at a disadvantage compared to the other controllers presented in this paper. It is highly unlikely that any improvement can be made on the neural network using an approach that is similar to the one presented in this paper. 

It is very likely, however, that significant improvements can be if the neural network is used to predict future game states or if simulations of the game state is used as inputs to the neural network. This was not experimented with, as the intention was to test how well a neural network could classify the simple inputs into move actions.

\subsection{Monte Carlo Tree Search controller}
\begin{table}[h]
\caption{MCTS controller results}
\begin{center}
\begin{tabular}{|c|c|c|c|c|c|}
\hline
\multicolumn{1}{|c|}{\raisebox{-0.50ex}[0cm][0cm]{\!Ghost\!}}
& \multicolumn{1}{|c|}{Average}
& \multicolumn{1}{|c|}{Max}
& \multicolumn{1}{|c|}{Min} 
& \multicolumn{1}{|c|}{Standard}
& \multicolumn{1}{|c|}{Relative}\\
controller & score & score & score & deviation & std dev \\ \hline
Legacy2 &  $6688.45$  &  $23450$  &  $1620$ & $2846.99$ & $42.57\%$\\ \hline
Starter &  $3902.24$  &  $13100$  &  $1900$ & $1647.25$ & $42.21\%$\\ \hline
Aggressive &  $7291.5$  &  $23530$  &  $1790$ & $3214.40$ & $44.08\%$\\ \hline
Random &  $10938.02$  &  $28820$  &  $2400$ & $6103.35$ & $55.80\%$\\ \hline
\end{tabular}
\label{table-mctsResults}
\end{center}
\end{table}

The scores of the Ms. PacMan controller presented in this paper is nowhere near the level other controllers\cite{pepels2012enhancements}, but it is also not nearly as advanced. Because the controller is simulated using the Legacy2 ghosts, it is not surprising that the controller performs better against Legacy2 ghosts than against the Starter ghosts. Due to the predictive nature of the simulations, it is not surprising that the relative standard deviation against the Random ghost controller is much higher than the other three ghost controllers.

\subsubsection{Suggested improvements}
A better evaluation of both long term goals and end game tactics, such as suggested by \citet{pepels2012enhancements}, would go a long way to improve the performance of the controller. Currently PacMan only evaluates the survival rate and pill score of the simulated playthroughs. However, if no pill is within range of PacMan, such as when there only a few pills left in the maze, the only goal of the controller is to stay alive. This can, in some cases, result in an endless dodging of ghosts. Checking if PacMan is boxed in by ghosts in addition to whether he was eaten in the last time step could lead to a more accurate prediction of the survival rate of a node.

Additionally, more experimentation of the exploration and max path to root variables could be done in order to further increase the performance of the controller. This could done either manually or using an evolutionary algorithm. A better fitness function than running trials is likely required, however, as the trials take a long time to run for the controller.

\subsection{Behaviour Tree with Evolutionary Algorithm controller}
\begin{table}[h]
\caption{Behaviour Tree (EA) controller results}
\begin{center}
\begin{tabular}{|c|c|c|c|c|c|}
\hline
\multicolumn{1}{|c|}{\raisebox{-0.50ex}[0cm][0cm]{\!Ghost\!}}
& \multicolumn{1}{|c|}{Average}
& \multicolumn{1}{|c|}{Max}
& \multicolumn{1}{|c|}{Min} 
& \multicolumn{1}{|c|}{Standard}
& \multicolumn{1}{|c|}{Relative}\\
controller & score & score & score & deviation & std dev \\ \hline
Legacy2 &  $10377.07$  &  $30840$  &  $2420$ & $5054.37$ & $48.71\%$\\ \hline
Starter &  $10446.4$  &  $32180$  &  $2640$ & $5474.55$ & $52.41\%$\\ \hline
Aggressive &  $24939.82$  &  $99400$  &  $3580$ & $15330.57$ & $61.47\%$\\ \hline
Random &  $59324.13$  &  $98420$  &  $5450$ & $18644.56$ & $31.43\%$\\ \hline
\end{tabular}
\label{table-bteaResults}
\end{center}
\end{table}

As shown in table \ref{table-bteaResults}, the relative standard deviation of the trials against the random ghosts is much lower than against the other three ghost controllers. This is similar to the results of the other behaviour tree controller (shown in table \ref{table-compResults}). Comparing the relative standard deviation of the results of the two controllers, it is the same against Starter and Aggressive ghosts, lower against Legacy2 ghosts and higher against Random ghosts. The average score of the evolved controller is higher against all ghost controllers, except against the standard ghosts. It is not surprising that performance has been improved against the Legacy2 ghosts, as the fitness of individuals in the evolutionary algorithm was determined using the Legacy2 controller. It is surprising that there is a setback against the Starter ghosts, even though the setback is small.

\subsubsection{Suggested improvements}
With the current structure of the behaviour tree itself, it is unlikely that it will perform much better than it currently does, due to the results from running the evolutionary algorithm. Any change to the structure of the behaviour tree, both in the addition of new behaviours or the removal of any current behaviours, is likely to completely change the performance of the controller. A more advanced tree will likely perform better, as it will be able to make more advanced decisions.

Several improvements can be made to the evolutionary algorithm. The first priority should be to experiment with adding recombination as a variation operator. More experimentation with different population amounts, parents and new children parameters is also likely to produce better results, if only slightly. It is also likely that a different representation of the genotype would allow for more complex recombination and mutation methods. Adding age as a consideration when doing the survivor selection process could also help in getting out of a local optimum.